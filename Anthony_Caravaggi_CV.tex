\documentclass[]{article}
\usepackage{lmodern}
\usepackage{amssymb,amsmath}
\usepackage{ifxetex,ifluatex}
\usepackage{fixltx2e} % provides \textsubscript
\ifnum 0\ifxetex 1\fi\ifluatex 1\fi=0 % if pdftex
  \usepackage[T1]{fontenc}
  \usepackage[utf8]{inputenc}
  \usepackage{eurosym}
\else % if luatex or xelatex
  \ifxetex
    \usepackage{mathspec}
  \else
    \usepackage{fontspec}
  \fi
  \defaultfontfeatures{Ligatures=TeX,Scale=MatchLowercase}
  \newcommand{\euro}{€}
\fi
% use upquote if available, for straight quotes in verbatim environments
\IfFileExists{upquote.sty}{\usepackage{upquote}}{}
% use microtype if available
\IfFileExists{microtype.sty}{%
\usepackage{microtype}
\UseMicrotypeSet[protrusion]{basicmath} % disable protrusion for tt fonts
}{}
\usepackage[margin=1in]{geometry}
\usepackage{hyperref}
\hypersetup{unicode=true,
            pdfborder={0 0 0},
            breaklinks=true}
\urlstyle{same}  % don't use monospace font for urls
\usepackage{graphicx,grffile}
\makeatletter
\def\maxwidth{\ifdim\Gin@nat@width>\linewidth\linewidth\else\Gin@nat@width\fi}
\def\maxheight{\ifdim\Gin@nat@height>\textheight\textheight\else\Gin@nat@height\fi}
\makeatother
% Scale images if necessary, so that they will not overflow the page
% margins by default, and it is still possible to overwrite the defaults
% using explicit options in \includegraphics[width, height, ...]{}
\setkeys{Gin}{width=\maxwidth,height=\maxheight,keepaspectratio}
\IfFileExists{parskip.sty}{%
\usepackage{parskip}
}{% else
\setlength{\parindent}{0pt}
\setlength{\parskip}{6pt plus 2pt minus 1pt}
}
\setlength{\emergencystretch}{3em}  % prevent overfull lines
\providecommand{\tightlist}{%
  \setlength{\itemsep}{0pt}\setlength{\parskip}{0pt}}
\setcounter{secnumdepth}{0}
% Redefines (sub)paragraphs to behave more like sections
\ifx\paragraph\undefined\else
\let\oldparagraph\paragraph
\renewcommand{\paragraph}[1]{\oldparagraph{#1}\mbox{}}
\fi
\ifx\subparagraph\undefined\else
\let\oldsubparagraph\subparagraph
\renewcommand{\subparagraph}[1]{\oldsubparagraph{#1}\mbox{}}
\fi

%%% Use protect on footnotes to avoid problems with footnotes in titles
\let\rmarkdownfootnote\footnote%
\def\footnote{\protect\rmarkdownfootnote}

%%% Change title format to be more compact
\usepackage{titling}

% Create subtitle command for use in maketitle
\newcommand{\subtitle}[1]{
  \posttitle{
    \begin{center}\large#1\end{center}
    }
}

\setlength{\droptitle}{-2em}

  \title{}
    \pretitle{\vspace{\droptitle}}
  \posttitle{}
    \author{}
    \preauthor{}\postauthor{}
    \date{}
    \predate{}\postdate{}
  

\begin{document}

\section{Dr.~Anthony Caravaggi}\label{dr.anthony-caravaggi}

University of South Wales, Upper Glyntaff Campus, 9 Graig Fach,
Pontypridd CF37 4BB\\
\textbf{E-mail} \url{ar.caravaggi@gmail.com} \textbar{} \textbf{Website}
\url{http://arcaravaggi.github.io} \textbar{} \textbf{Twitter}
@\href{https://twitter.com/thonoir}{thonoir}\\
\textbf{GitHub}
\href{https://arcaravaggi.github.io}{arcaravaggi.github.io} \textbar{}
\href{https://scholar.google.co.uk/citations?user=PXf81KIAAAAJ\&hl=en}{Google
Scholar}

\subsubsection{Career History}\label{career-history}

\textbf{2019-} Lecturer in Natural History,
\href{http://www.southwales.ac.uk}{University of South Wales}\\
\textbf{2017-19} Postdoctoral researcher;
\href{https://www.ucc.ie/en/forestecology/research/shine/}{Supporting
Hen harriers in Novel Environments (SHINE)}\\
School of Biological, Earth and Environmental Sciences,
\href{http://www.ucc.ie/en/}{University College Cork}\\
\textbf{2017-18} Visiting Research Fellow; movement ecology, invasive
species\\
School of Biological Sciences,
\href{http://www.qub.ac.uk/schools/SchoolofBiologicalSciences/}{Queen's
University Belfast}

\subsubsection{Education}\label{education}

\textbf{2016} PhD conservation biology/invasion ecology,
\href{http://www.qub.ac.uk/schools/SchoolofBiologicalSciences/}{Queen's
University Belfast}\\
\textbf{2011} MRes Biodiversity and Conservation,
\href{https://www.leeds.ac.uk/}{University of Leeds}\\
\textbf{2009} BSc Zoology with Conservation (Hons),
\href{https://www.bangor.ac.uk/}{Bangor University}

\subsubsection{Professional activities}\label{professional-activities}

\textbf{Associate Editor}: Remote Sensing in Ecology and Conservation

\textbf{Peer reviewer}: African Journal of Ecology; Austral Ecology;
Biological Conservation; Bird Study; Canadian Journal of Zoology;
Diversity and Distributions; Global Change Biology; Journal of Natural
History; Journal of Zoology; Mammal Communications; Mammal Research;
Mammal Review; PeerJ; PLOS One; Remote Sensing in Ecology and
Conservation; The European Zoological Journal; Zoo Biology.

\textbf{Funding review}: British Ecological Society Grant Review
College; Croatian Science Foundation.

\textbf{Consultant}: The Hen Harrier Reintroduction Project (University
of Exeter and Natural England).

\subsection{Research}\label{research}

I employ field and laboratory-based experiments, along with archival
data, to answer specific questions relating to the ecology of species
and communities to inform conservation, management and policy-related
processes. My current work includes an analysis of mammal activity
patterns, landscape-scale habitat and distribution modelling,
quantification of the impacts of non-native rodents on insular seabirds
and an assessment of wildlife management practises.

\textbf{Research interests}; Conservation biology; urban ecology;
invasive species; population \& community ecology; movement \& spatial
ecology; rare and endangered species; sustainability; rewilding; citizen
science; remote sensing.

\subsection{Peer-Reviewed
Publications}\label{peer-reviewed-publications}

2018 \textbf{Caravaggi A}, Vallely M-C, Hogg K, Freeman M, Fadaei E,
Reid N, Tosh D. Seasonal and predator-prey effects on circadian activity
of free-ranging mammals revealed by camera traps. \emph{PeerJ}. 6:e5827.
doi: \href{https://doi.org/10.7717/peerj.5827}{10.7717/peerj.5827}

\textbf{Caravaggi A}, Cuthbert RJ, Ryan RG, Cooper J, Bond AL. The
cumulative impacts of introduced house mice on the breeding success of
nesting seabirds on Gough Island. \emph{Ibis}. doi:
\href{https://onlinelibrary.wiley.com/doi/10.1111/ibi.12664}{10.1111/ibi.12664}

\textbf{Caravaggi A}, Plowman A, Wright D, Bishop C. The composition of
captive ruffed lemur (\emph{Varecia} spp.) diets in UK zoological
collections, with reference to the problems of obesity and iron storage
disease. \emph{Journal of Zoo Animal Research}. 6: 41-49. doi:
\href{https://doi.org/10.19227/jzar.v6i2.301}{10.19227/jzar.v6i2.301}

Havlin P, \textbf{Caravaggi A}, Montgomery WI. The diet and abundance of
an introduced insular population of Red-Necked Wallabies, \emph{Macropus
rufogriseus}. \emph{Canadian Journal of Zoology}. 96: 357-365. doi:
\url{10.1139/cjz-2017-0090}

2017\\
O'Halloran J, Kelly TC, Quinn JL, Irwin S, Fernandez-Bellon D,
\textbf{Caravaggi A}, Smddy P. Current ornithological research in
Ireland: seventh Ornithological Research Conference, UCC, November 2017.
\emph{Irish Birds}. 10: 598-638.

\textbf{Caravaggi A}, Montgomery WI, Reid N. Management and control of
invasive brown hares (\emph{Lepus europaeus}): contrasting attitudes of
selected environmental stakeholders and the wider rural community.
\emph{Proceedings of the Royal Irish Academy: Biology \& Environment}.
117B: 1-11. doi:
\href{http://www.jstor.org/stable/10.3318/bioe.2017.08}{10.3318/bioe.2017.08}

Tennant JP, Graziotin D, Jacques DC, Waldner F, Dugan JM, Mietchen D,
Elkhatib Y, Collister LB, Pikas CK, Crick T, Masuzzo P,
\textbf{Caravaggi A}, \emph{et al.} A multi-disciplinary perspective on
emergent and future innovations in peer review. \emph{F1000Research}.
6:1151. doi:
\href{https://f1000research.com/articles/6-1151/v3}{10.12688/f1000research.12037.3}

\textbf{Caravaggi A}, Banks P, Burton C, Finlay CMV, Hayward M, Haswell,
PM, Rowcliffe JM, Wood M. A review of camera trapping for conservation
behaviour research. \emph{Remote Sensing in Ecology and Conservation}.
doi:
\href{http://onlinelibrary.wiley.com/doi/10.1002/rse2.48/abstract}{10.1002/rse2.48}

\textbf{Caravaggi A}. remBoot: An R package for Random Encounter
Modelling. \emph{Journal of Open Source Software}. 2(10). doi:
\href{http://joss.theoj.org/papers/10.21105/joss.00176}{10.21105/joss.00176}

\textbf{Caravaggi A}, Leach K, Santilli F, Rintala J, Helle P, Tiainen
J, Bisi F, Martinoli A, Montgomery WI, Reid N. Niche overlap of mountain
hare subspecies and the vulnerability of their ranges to invasion by the
European hare; the (bad) luck of the Irish. \emph{Biological Invasions}.
19(2): 655-674. doi:
\href{http://link.springer.com/article/10.1007/s10530-016-1330-z}{10.1007/s10530-016-1330-z}

2016\\
\textbf{Caravaggi A}, Zaccaroni M, Riga F, Schai-Braun SC, Dick JTA,
Montgomery WI, Reid N (2016) An invasive-native mammalian species
replacement process captured by camera trap survey Random Encounter
Models. \emph{Remote Sensing in Ecology and Conservation}. 2: 45-58.
doi:
\href{http://onlinelibrary.wiley.com/doi/10.1002/rse2.11/abstract}{10.1002/rse2.11}

2015\\
\textbf{Caravaggi A}, Montgomery WI, Reid N et al. (2015) Range
expansion and comparative habitat use of insular, congeneric lagomorphs:
invasive European hares \emph{Lepus europaeus} and endemic Irish hares
\emph{Lepus timidus hibernicus}. \emph{Biological Invasions}. 17(2):
687-698. doi:
\href{http://link.springer.com/article/10.1007/s10530-014-0759-1}{10.1007/s10530-014-0759-1}

\begin{quote}
\emph{In review}
\end{quote}

\textbf{Caravaggi A}, Irwin S, Lusby J, Ruddock M, O'Toole L, Mee A,
Nagle T, O'Neill S, Tierney D, O'Halloran J. Factors affecting territory
site selection and breeding parameters of Hen Harriers, \emph{Circus
cyaneus}, in Ireland. \emph{Ibis}.

\textbf{Caravaggi A}, Irwin S, Lusby J, Ruddock M, O'Toole L, Mee A,
Nagle T, O'Neill S, Tierney D, O'Halloran J. Anthropogenic pressures on
Hen Harrier breeding habitat. \emph{Bird Conservation International}.

\begin{quote}
\emph{Book chapters}
\end{quote}

2018\\
\textbf{Caravaggi A}. Lagomorpha Life History. In: \emph{Encyclopedia of
Animal Cognition and Behavior}, Vonk J, Shackelford TK (eds.) doi:
\href{https://link.springer.com/referenceworkentry/10.1007/978-3-319-47829-6_1206-1}{10.1007/978-3-319-47829-6\_1206-1}

\textbf{Caravaggi A}. Lagomorpha Navigation. In: \emph{Encyclopedia of
Animal Cognition and Behavior}, Vonk J, Shackelford TK (eds.) doi:
\href{https://link.springer.com/referenceworkentry/10.1007/978-3-319-47829-6_1164-1}{https://doi.org/10.1007/978-3-319-47829-6\_1164-1}

\begin{quote}
\emph{Technical reports}
\end{quote}

2018\\
McGowan N, Dingerkus K, Stone R, \textbf{Caravaggi A}, et al. Hare
Survey of Ireland 2018/2019: Interim Report (June) 2018. Report produced
for the Department of Arts, Heritage and the Gaeltacht.

\begin{quote}
\emph{Other publications}
\end{quote}

2017 \textbf{Caravaggi A}, James K. Twittersphere: Conferencing in 140
characters. \emph{Nature} {[}Correspondence{]} 549: 458. doi:
\href{http://www.nature.com/nature/journal/v549/n7673/full/549458d.html}{doi:10.1038/549458d}

\subsection{Teaching}\label{teaching}

As a teacher I aim to provide my students with skills, knowledge and
confidence critical to their academic and personal development. I take
an inclusive approach, developing strong relationships and encouraging
discussion.

\begin{quote}
\emph{Qualifications}
\end{quote}

2018\\
Postgraduate Certificate in Teaching and Learning in Higher Education,
University of London (\emph{ongoing}).

\begin{quote}
\textbf{Module development}
\end{quote}

2017\\
\emph{BL6024 Quantitative skills for biologists using R}\\
Quantitative skills required by postgraduate students to successfully
conduct and publish their research, with a focus on data analysis and
graphing, statistics, and basic modelling, as generally implemented by
zoologists and ecologists. Supported by a new R Peer Group mailing list
and pop-up clinics.

\begin{quote}
\textbf{Teaching}
\end{quote}

2019 Biodiversity \& biogeography\\
MSc research projects\\
Research methods in natural history\\
Terrestrial \& aquatic conservation\\
2018 BL6024 Quantitative skills for biologists using R. Conservation
biology. 2017 BL6024 Quantitative skills for biologists using R.\\
2015 Introduction to GIS (\emph{Teaching Assistant}).\\
Animal Biology (\emph{Teaching Assistant}).\\
Undergraduate residential field course.\\
2014 Undergraduate residential field course.

\begin{quote}
\emph{Field-based teaching}
\end{quote}

2016 South American rainforest biodiversity and conservation
(Ecuador).\\
2015 Undergraduate residential field course, Northern Ireland.\\
2014 Undergraduate residential field course, Northern Ireland.\\
Bird survey techniques.

\begin{quote}
\textbf{Project supervision}
\end{quote}

2018\\
PhD. Samantha Ball, \emph{Managing hares at Dublin airport}. University
College Cork.\\
Erasmus+. Enrique Da Paz, \emph{Hen Harrier distribution under a
changing climate}, University College Cork/University of Barcelona.

2017-19\\
MRes. Alan McCarthy, \emph{Top-down and bottom-up processes and Hen
Harriers in Ireland}. University College Cork\\
2017\\
BSc. Chloe O'Mahoney, \emph{Regional variation in Hen Harrier chick
growth rates}. University College Cork.

2015\\
MSc. Ashley Irwin, \emph{Surveying otters along an urban-rural gradient
using camera traps}. Queen's University Belfast.\\
MSc. Lauren Finley, \emph{Estimaing hare population densities using
camera traps}. Queen's University Belfast.\\
BSc. Paige Havlin, \emph{Population size, habitat associations and diet
of wallabies on the Isle of Man}. Queen's University Belfast.\\
Scientists Without Borders. Lazaro Silva, \emph{Camera trapping for
hares}. Queen's University Belfast.

2014\\
BSc. Mengyi Liu, \emph{Behaviour and stress in sympatric hare species}.
Queen's University Belfast.\\
BSc. Claire Molloy, \emph{Ethology of hares in Ireland} Queen's
University Belfast.

2013\\
BSc. Ashley Irwin, \emph{The nutritional composition of vegetation
associated with hares}. Queen's University Belfast.

\begin{quote}
\textbf{Instruction and demonstration}
\end{quote}

2016-17 Course tutor: ornithology, camera traps, social media.
Eco-Explore, Cardiff\\
Developed materials and taught courses on UK bird identification, the
use of camera traps in ecology, and social media for scientists

2016 Lecturer and field ornithologist. Operation Wallacea, Ecuador\\
Co-developed and delivered a 5-part series of lectures focussing on
rainforest biodiversity and conservation, and conducted bird ringing
studies including demonstrating techniques and training students in
biometric data gathering. Also trained an assistant on mist nest
extraction and data recording

2015 Otter surveys, MSc, Queen's University Belfast\\
Trained a Master's student in otter survey techniques and field signs,
and advised on camera trap survey methodology

\subsection{Fieldwork experience}\label{fieldwork-experience}

2018 Camera trapping, small mammal trapping, bird point counts. Upland
areas.\\
2017 Hen Harrier nest surveys. Upland areas.\\
Breeding Bird Survey (BTO). Upland areas; windfarm-adjacent.\\
2016 Bird, bat and reptile surveys and camera trapping. South American
rainforest.\\
2015 Camera trapping for otters. Rivers and riparian areas.\\
2014 Badger sett surveys. Broadleaved woodlands.\\
2012-15 Hare surveys (distance sampling, camera traps, fecal). Uplands
and agricultural.\\
2013 Coastal biodiversity surveys. Tidal pools.\\
2012 Wader and gull surveys. Tidal flats.\\
2011 Small mammal trapping. Agricultural fields.\\
Reptile surveys. Clearfell areas in coniferous forests.\\
Bird point count, reptile, mammal and invertebrate surveys. Marsh and
fen.\\
Bat roost surveys (visual and detector). Urban areas.\\
Great crested newt surveys. Agricultural areas.\\
Bird surveys. Broadleaved woodland.\\
2009-11 Breeding Bird Surveys (BTO). Upland areas.\\
2007-18 Bird ringing (mist, whoosh and cannon nets and traps). Woodland,
fen, marsh, grassland, coastal and agricultural areas.\\
2007-09 Pine marten surveys. Coniferous woodland.\\
2007 Passerine bird and invertebrate surveys. Shrub and broadleaved
woodland.\\
Moth trapping. Treborth Botanic gardens.\\
2006-08 Bird surveys and nestbox monitoring. Broadleaved woodland.

\subsection{Fellowships, funding and
Awards}\label{fellowships-funding-and-awards}

2018 Funded postgraduate programme (PhD), \euro{}96,000, Irish Research
Council\\
2017 Project tender, \euro{}174,000, National Parks and Wildlife
Service\\
2015 Emily Sarah Montgomery Travel Scholarship, £200, Queen's University
Belfast\\
2015 Training and Travel grant, £500, British Ecological Society\\
2012 Research grant, £2,000, People's Trust for Endangered Species\\
2012 NIEA funded PhD studentship, £82,720, Queen's University Belfast\\
2010 NERC Master's scholarship, £7,500, University of Leeds

\subsection{Presentations}\label{presentations}

\begin{quote}
\textbf{Conferences} (\emph{oral presentations})
\end{quote}

12 \textbf{Caravaggi A}, Montgomery WI, Reid N (2017) An invasion in
progress: Impacts of the non-native European hare on the native Irish
hare. International Congress for Conservation Biology, Cartagena

11 \textbf{Caravaggi A} (2017) The changing face of UK conservation.
Mammal Society Student Conference (\emph{invited keynote})

10 \textbf{Caravaggi A} (2017) University Mammal Challenge: distributed
mammal surveys with student scientists. Mammal Society Spring
Conference, Cambridge

9 \textbf{Caravaggi A}, Montgomery WI, Reid N (2016) Invasion ecology:
quantifying the impact of non-native European hares on the endemic Irish
hare. British Ecological Society Annual Meeting, Liverpool

8 \textbf{Caravaggi A}, Hayward M (2016) People and hedgehogs in Wales.
North Wales Mammal Symposium, Bangor

7 \textbf{Caravaggi A}, Montgomery WI, Reid N (2016) Invasion ecology:
quantifying the impact of non-native European hares on the endemic Irish
hare. Mammal Society Spring Conference, Yarnfield

6 \textbf{Caravaggi A} 2015 Quantifying endemic uniqueness and
ecological equivalency between native and invasive leporids. 7th
European Congress of Mammalogy, Stockholm, Sweden

5 \textbf{Caravaggi A}, Leach K, Santilli F, Rintala J, Helle P, Tiainen
J, Bisi F, Martinoli A, Montgomery WI, Reid N (2015) Democratisation of
invasive species interventions: assessing attitudes towards lethal
control of mammalian invaders. Mammal Society Spring Conference,
Lancaster

4 \textbf{Caravaggi A}, Zaccaroni M, Riga F, Schai-Braun SC, Dick JTA,
Montgomery WI, Reid N (2015) Using the Random Encounter Model to
estimate spatial patterns in invasive and native hare densities in
allopatry and sympatry. Mammal Society Student Conference, Lancaster

3 \textbf{Caravaggi A}, Montgomery WI, Reid N (2014) Using the Random
Encounter Model to estimate spatial patterns in invasive and native hare
densities in allopatry and sympatry. Mammal Society Spring Conference,
Birmingham

2 \textbf{Caravaggi A}, Montgomery WI, Reid N (2013) Distribution and
differential habitat associations of invasive European and endemic Irish
hares. 11th International Mammal Congress, Belfast

1 \textbf{Caravaggi A}, Montgomery WI, Reid N (2013) Invasion ecology:
quantifying the impact of non-native European hares on the endemic Irish
hare. Mammal Society Student Conference, Staffordshire

\begin{quote}
\textbf{Conferences} (\emph{posters})
\end{quote}

9 \textbf{Caravaggi A}, Vallely M-C, Hogg K, Freeman M, Fadaei E, Reid
N, Tosh D (2017) Activity and temporal relationships of free-ranging
mammals revealed by camera traps. Mammal Society Spring Conference,
Cambridge

8 \textbf{Caravaggi A}, Montgomery WI, Reid N (2015) The democratisation
of conservation: public attitudes towards lethal control of invasive
leporids and implications for species management. 7th European Congress
of Mammalogy, Stockholm, Sweden

7 \textbf{Caravaggi A}, Zaccaroni M, Riga F, Schai-Braun SC, Dick JTA,
Montgomery WI, Reid N (2015) Optimising camera trap Random Encounter
Model survey effort: A case study using an invasive-native species
replacement process. Vth International Wildlife Management Congress,
Sapporo, Japan

6 Beresford NA, Wood MD, Gaschack S, Gulyaichenko EA, \textbf{Caravaggi
A} (2015) Use of wildlife camera traps to within the Chernobyl Exclusion
Zone. Mammal Society Spring Conference, Lancaster

5 \textbf{Caravaggi A}, Montgomery WI, Reid N (2013) Estimating hare
numbers using camera trapping and the Random Encounter Model. 11th
International Mammal Congress, Belfast

4 \textbf{Caravaggi A}, Montgomery WI, Reid N (2012) Invasion ecology:
quantifying the impact of non-native European hares on the endemic Irish
hare. 2nd All Ireland Mammal Symposium (AIMS), Dublin

3 \textbf{Caravaggi A}, Montgomery WI, Reid N (2012) Invasion ecology:
quantifying the impact of non-native European hares on the endemic Irish
hare. 4th World Lagomorph Conference, BOKU, Vienna

2 \textbf{Caravaggi A}, Hughes W (2011) Patrilineal inheritance of
grooming behaviour and parasite resistance in leaf-cutting ants
(Acromyrmex sp.). Postgraduate Student Forum, University of Leeds

1 \textbf{Caravaggi A}, Bishop, C (2009) Dietary and faecal iron levels
in captive black-and-white ruffed lemurs (Varecia spp.). 11th BIAZA
Research Symposium, Blackpool

\begin{quote}
\textbf{Invited seminars}
\end{quote}

2014 Distribution and differential habitat associations of invasive
European and endemic Irish hares. Bangor University

\subsection{Conference organisation}\label{conference-organisation}

2018 2nd Biotweeps Twitter Conference, \#BTCon18 (\emph{principal
organiser}).\\
2017 7th Irish Ornithological Research Conference (\emph{scientific
committee}).\\
2017 Biotweeps Twitter Conference, \#BTCon17 (\emph{conception and
organisation}).\\
2014-18 Mammal Society Student Conference (\emph{principal organiser}).

\subsection{Personal development}\label{personal-development}

\begin{quote}
\textbf{Training and workshops}
\end{quote}

2018 Postgraduate Certificate in Teaching and Learning in Higher
Education.\\
Unconscious bias awareness.\\
Developing Horizon 2020 grant applications.\\
2017 Postgraduate Certificate in Teaching and Learning in Higher
Education, University College Cork.\\
NERC Advanced Training, Systematic Review and Meta-Analysis, University
of East Anglia.\\
Level 3 Teaching Assistant, E-Training Hub.\\
Public Outreach, Science Made Simple.\\
2016 Mentor Training, British Ecological Society.\\
Conservation Ecology Special Interest Group workshop, British Ecological
Society Annual Meeting.\\
Acoustic Analysis, North Wales Mammal Symposium.\\
2014 Using R, Queen's University Belfast.\\
2012 Lagomorph Systematics, 4th World Lagomorph Conference.

\begin{quote}
\textbf{Licenses and Certifications}
\end{quote}

2018 LAST Ireland (animal welfare in research; License no: 09-2018)\\
2014 ASPA laboratory animal handling license (TSA/47/14)\\
2014 Emergency first aid; field first aid\\
2009 Bird ringing `C' license holder, British Trust for Ornithology\\
2008 Marine mammal medic, British Divers Marine Life Rescue\\
2002 Full manual UK driving license, held without incident or penalty

\begin{quote}
\textbf{Societal memberships}
\end{quote}

(since)\\
2017 British Ornithologist's Union\\
2015 South Wales Mammal Group; Welsh Ornithological Society\\
2013 Society of Conservation Biology (SCB)\\
2012 The Mammal Society; British Ecological Society\\
2011 Royal Society of Biology\\
2009 The British Trust for Ornithology

\begin{quote}
\textbf{Organisational and Working Groups}
\end{quote}

(since)\\
2017 SCB Participatory and Citizen Science\\
2016 South Wales Mammal Group (\emph{committee})\\
2016 British Ecological Society Review College\\
2013 SCB Urban Ecology

\begin{quote}
\textbf{Additional roles}
\end{quote}

2018-19 UCC BEES Education and Public Engagement Committee.\\
2016-18 University Mammal Challenge, The Mammal Society (coordinator).\\
2016 Community Ecology session chair, British Ecological Society Annual
Meeting.\\
Advisor to National Geographic on red fox ecology.\\
British Ecological Society Women in Ecology (WIE) mentor.\\
Mammal (hares) consultant for The Mammal Society.\\
2013 Wildlife consultant, Biological records for the North East of
Scotland (NESBReC).\\
2013-17 UK STEM Ambassador.\\
2012-15 EBEE postgraduate representative, Queen's University Belfast.\\
2012-15 SBS Athena SWAN committee, Queen's University Belfast.\\
2007-09 Undergraduate student representative, Bangor University
Biological Sciences.\\
Founding committee \& publicity officer, Bangor University Zoological
Society.

\begin{quote}
\textbf{Science communication}
\end{quote}

I am a strong supporter of science communication and outreach. I believe
that it is important to encourage an interest in science and the natural
world in people of all ages. Moreover, the success or failure of many
modern conservation objectives increasingly require a connection with
and understanding of the public. I have given talks to over 20 school,
local interest, community and natural history groups in the UK, USA and
Canada, in-person and via video calls.

\emph{Contributions}\\
2018 UCC BEES Committee for Education and Public Engagement.\\
BBC Wildlife magazine feature on wallabies (\emph{invited expert}).\\
I'm a Scientist Get Me Out of Here! Careers Zone expert.\\
2017 Sanger Institute 25 Genomes (\emph{species champion - St.~Kilda
wren}).\\
Culture Night, University College Cork.\\
Blog post - A brief history of an Early Career Researcher, Errant
Science.\\
Conservation \& ecology expert on the sub-reddit /r/AskScience.\\
Blog post - Rejected by the Rainforest, I F-ing Love Conservation.\\
Science communication article, CBS (\emph{\#bestcarcass,
contributor}).\\
Rabaiotti D, Caruso N 2017. Quercus books, ISBN: 9781786488275
(\emph{contributor}).\\
Student Science Partnerships, USA.\\
Bird song workshop and ringing demonstration, Parc Slip Nature Reserve,
Wales.\\
2016-18 Skype a Scientist (\emph{5 classrooms across the US}).\\
2016 Blog post - Camera traps reveal a hare-raising situation, RSEC
blog, ZSL.\\
Blog post - On closer inspection, it looks like a hare transplant,
QUBio.\\
Magazine article - Is there intelligent life on Earth? Save our Souls
magazine (\emph{author}).\\
Letters to a Pre-scientist (\emph{correspondent}).\\
2015 Blog post - Putting wildlife in the frame, QUBio.\\
Blog post - Communities and conservation in developing countries,
QUBio.\\
Nature News (\emph{invited comment}).\\
Bird ringing demonstration, Parc Slip Nature Reserve, Wales.\\
Blog post - Dear Alice, the hare might be sane after all, QUBio.\\
Blog post - What's in a name? QUBio.\\
2014-16 Various Local Mammal Group newsletters (\emph{invited
author}).\\
Created @Biotweeps, a science communication Twitter project
(\textgreater{}16.5k followers).\\
Multi-school careers advice and speed-networking event (\emph{invited
expert}).\\
2013-18 I'm a Scientist Get Me Out of Here! Enquiry Zone \& Careers
Zone.\\
2013-17 UK STEM Ambassador.\\
2013 Science outreach event, Techniquest, Cardiff (\emph{presenter for
RSPB}).

\begin{quote}
\textbf{Media engagements}
\end{quote}

\emph{Radio}

2018 BBC Radio Wales, UK.\\
BBC Radio Derby, UK.\\
MostlyScience live (\emph{podcast}).\\
2017 BBC Coventry \& Warwickshire, Wallabies in the UK.\\
The Edgar Ortega Radio Show, North America (\emph{podcast}).\\
2016 SciComm with Dr.~Mike (\emph{podcast, 2 shows}).\\
2015 BBC Radio Ulster, Hares at Belfast airport.\\
BBC News School Report for St.~Pius X, Magherafelt's Biodiversity
Club.\\
BBC Radio Ulster, \emph{Good Morning Ulster}, Invasive species.\\
2013 RTE Radio 1, \emph{Mooney Goes Wild}, Hares in Ireland.

\emph{Television}

2015 Countryfile, BBC\\
2013 River Bann, RTE Television

\emph{Other media}

2018 BBC Wildlife (invited article).\\
2017 The Washington Post (\emph{interview}).\\
2016 National Geographic advisor (\emph{mammal expert}).\\
Biotechin.asia (\emph{interview}).\\
2015 Media battle highlights importance of scientific credit.
ScienceNews (\emph{interview}).\\
Astrotweeps and Biotweeps: rotating Twitter accounts. Crastina.se
(\emph{interview}).\\
BBC News (\emph{interview and featured article}).\\
A call for beautiful prose in papers. Nature News (\emph{Interview}).\\
2014 Species profiles: Mountain hare and Irish hare, The Mammal
Society.\\
2013 `Hare' raising project in Mid-Ulster, Mid Ulster Mail
(\emph{Interview}).

\subsection{Skills}\label{skills}

\begin{quote}
\emph{Analytical sotware}
\end{quote}

R, SPSS, Minitab, Distance, MARK, ArcGIS, QGIS, MaxEnt, NetLogo, MAVIS
NVC, Patch Analyst, GeneMapper, bespoke software.

\begin{quote}
\emph{Additional tools}
\end{quote}

Microsoft Office, GitHub, RMarkdown, graphics packages (e.g.~Adobe
creative suite), website creation.

\begin{quote}
\emph{Statistical}
\end{quote}

General linear and mixed-effects models, Principal Components Analysis,
Species Distribution Modelling, Discriminant Function Analysis,
diversity indices, Mahalanobis and Euclidean distance, cross-correlation
functions, model averaging, Random Encounter Models, meta-analysis,
other parametric and non-parametric tests

\begin{quote}
\emph{Field}
\end{quote}

Off-road driving (4x4), multi-taxa (mammal, bird, invertebrate, reptile,
amphibian; UK coastal, marine and terrestrial) surveys (e.g.~mark and
recapture, distance sampling, camera traps, quadrats, scan/focal
sampling;) and identification, habitat surveys, animal handling
(amphibians, reptiles, invertebrates, and small-to-large birds and
mammals), orienteering

\begin{quote}
\emph{Laboratory}
\end{quote}

Multi-taxa dissection (mammal, bird, amphibian, invertebrate), species
identification via microscopy, ethological studies, tissue sampling, DNA
manipulation, electrophoresis, atomic absorption and colourimetric
spectroscopy, specimen preparation, soil chemical and biological
analysis

\newpage

\subsection{Referees}\label{referees}

\textbf{Prof.~John O'Halloran}\\
School of BEES, University College Cork, Distillery Field, N Mall,
Ireland\\
\textbf{Email}
\href{mailto:j.ohalloran@ucc.ie}{\nolinkurl{j.ohalloran@ucc.ie}}
\textbf{Tel} +353 (0)21 490 2407

\textbf{Dr.~Sandra Irwin}\\
School of BEES, University College Cork, Distillery Field, N Mall,
Ireland\\
\textbf{Email}: \href{mailto:s.irwin@ucc.ie}{\nolinkurl{s.irwin@ucc.ie}}
\textbf{Tel} +353 (0)21 490 4595

\textbf{Dr.~Neil Reid}\\
School of Biological Sciences, Medical Biology Centre, 97 Lisburn Road,
Belfast BT9 7BL\\
\textbf{Email}
\href{mailto:neil.reid@qub.ac.uk}{\nolinkurl{neil.reid@qub.ac.uk}}
\textbf{Tel} +44 (0)28 9097 2746

\textbf{Prof.~Ian Montgomery}\\
School of Biological Sciences, Medical Biology Centre, 97 Lisburn Road,
Belfast BT9 7BL\\
\textbf{Email}
\href{mailto:i.montgomery@qub.ac.uk}{\nolinkurl{i.montgomery@qub.ac.uk}}
\textbf{Tel} +44 (0)28 9097 2214

\textbf{Prof.~Jaimie Dick}\\
School of Biological Sciences, Medical Biology Centre, 97 Lisburn Road,
Belfast BT9 7BL\\
\textbf{Email}
\href{mailto:j.dick@qub.ac.uk}{\nolinkurl{j.dick@qub.ac.uk}}
\textbf{Tel} +44 (0)28 9097 2286

\textbf{Dr.~Tom Cameron}\\
School of Biological Sciences, University of Essex, Wivenhoe Park,
Colchester, CO4 3SQ\\
\textbf{Email}
\href{mailto:tcameron@essex.ac.uk}{\nolinkurl{tcameron@essex.ac.uk}}
\textbf{Tel} +44 (0)1206 874074


\end{document}
